
\section{Jointly Typical Sequences}
To calculate $Pe$ for jointly typical decoding, joining AEP is needed, and below is a brief explanation of the same.
For some joint distribution \( p(x_1, x_2, \ldots, x_k) \) where \((x_1, x_2, \ldots, x_k) \in \mathcal{X}_1 \times \mathcal{X}_2 \times \cdots \times \mathcal{X}_k\)  and let \((X_1, X_2, \ldots, X_k)\) denote a finite collection of discrete RV's.
Considering \(n\) independent copies of \(S\) where \(S\) is an ordered subset of these RV's.
\begin{equation}
    \Pr(S = s) = \prod_{i=1}^{n} \Pr(S_i = s_i), \quad s \in \mathcal{S}^n
\end{equation}
For such a subset \(S\) of RV's, By the law of large numbers:
\begin{equation}
    -\frac{1}{n} \log p(S_1, S_2, \ldots, S_n) = -\frac{1}{n} \sum_{i=1}^{n} \log p(S_i) \rightarrow H(S),
\end{equation}
for all \(2^k\) subsets, \(S \subseteq (X_1, X_2, \ldots, X_k)\), convergence happens  with probability 1 simultaneously.

\subsection{Some Definitions and Theorems}
%
\begin{tcolorbox}[boxrule=0pt,frame hidden,sharp corners,enhanced, opacityback=0, borderline west={2pt}{0pt}{red}]
\begin{defn}The set \( A^{(n)}_\epsilon (X_1, X_2, \ldots, X_k) \) is defined by
\begin{equation}
    A^{(n)}_\epsilon = \left\{ (x_1, x_2, \ldots, x_k) : \left| -\frac{1}{n} \log p(s) - H(S) \right| < \epsilon, \, \forall S \subseteq \{X_1, X_2, \ldots, X_k\} \right\}
\end{equation}
\end{defn}
\end{tcolorbox}
%
%
\begin{tcolorbox}[boxrule=0pt,frame hidden,sharp corners,enhanced, opacityback=0, borderline west={2pt}{0pt}{red}]
\begin{defn}for \(n\) sufficiently large notation \( a_n \doteq 2^{n(b \pm \epsilon)} \) to mean
\begin{equation}
    \left| \frac{1}{n} \log a_n - b \right| < \epsilon 
\end{equation}
\end{defn}
\end{tcolorbox}
%
%
\begin{tcolorbox}[boxrule=0pt,frame hidden,sharp corners,enhanced, opacityback=0, borderline west={2pt}{0pt}{blue}]
\begin{thm}For any \( \epsilon > 0 \), for sufficiently large \( n \),
\begin{enumerate}
    \item \( P(A^{(n)}_\epsilon(S)) \geq 1 - \epsilon, \quad \forall S \subseteq \{X_1, X_2, \ldots, X_k\}. \)
    \item \( s \in A^{(n)}_\epsilon(S) \Rightarrow p(s) \doteq 2^{-n(H(S) \pm \epsilon)}. \) 
    \item \( |A^{(n)}_\epsilon(S)| \doteq 2^{n(H(S) \pm 2\epsilon)}. \) 
    \item Let \( S_1, S_2 \subseteq \{X_1, X_2, \ldots, X_k\} \). If \((s_1, s_2) \in A^{(n)}_\epsilon(S_1, S_2)\), then
    \begin{equation}
        p(s_1, s_2) \doteq 2^{-n(H(S_1, S_2) \pm 2\epsilon)}
    \end{equation}
\end{enumerate}
\end{thm}
\end{tcolorbox}
%
%
\begin{tcolorbox}[boxrule=0pt,frame hidden,sharp corners,enhanced, opacityback=0, borderline west={2pt}{0pt}{blue}]
\begin{thm}Let \( S_1, S_2 \) be two subsets of \( X_1, X_2, \ldots, X_k \). For any
\( \epsilon > 0 \), define \( A^{(n)}_\epsilon(S_1 | S_2) \) to be the set of \( s_1 \) sequences that are jointly
\( \epsilon \)-typical with a particular \( s_2 \) sequence. If \( s_2 \in A^{(n)}_\epsilon(S_2) \), then for sufficiently large \( n \), we have
\begin{equation}
    |A^{(n)}_\epsilon(S_1 | S_2)| \leq 2^{n(H(S_1 | S_2) + 2\epsilon)},
\end{equation}
and
\begin{equation}
    (1 - \epsilon) 2^{n(H(S_1 | S_2) - 2\epsilon)} \leq \sum_{s_2} p(s_2) |A^{(n)}_\epsilon(S_1 | S_2)|
\end{equation}
\end{thm}
\end{tcolorbox}
%
%
\begin{tcolorbox}[boxrule=0pt,frame hidden,sharp corners,enhanced, opacityback=0, borderline west={2pt}{0pt}{blue}]
\begin{thm}Let \( A^{(n)} \) denote the typical set for the probability mass function \( p(s_1, s_2, s_3) \), and let

\begin{equation}
P(S_i = s_1, S_j = s_2; S_k = s_3) = \prod_{i=1}^{n} p(s_{i} | s_{j}) p(s_{j} | s_{k}) p(s_{k})
\end{equation}

Then

\begin{equation}
P\left((S_i, S_j, S_k) \in A^{(n)}\right) \approx 2^{-n(I(S_i; S_j | S_k) + I(S_i; S_k))}
\end{equation}
\end{thm}
\end{tcolorbox}
%
